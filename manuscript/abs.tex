\begin{abstract}
        Nuclear fuel cycle simulation scenarios are most naturally described as 
        constrained objective functions. The objectives are often systemic demands such as
        ``achieve 1\% growth for total electricity production and reach 10\% uranium
        utilization''. The constraints typically take the form of nuclear fuel cycle 
        and reactor technology availability (''reprocessing begins after 2025 
        and fast reactors first become available in 2050'').

        Fuel cycle simulation tools approach such objective functions in various ways.
        Some wrap realizations of the simulator in an external optimizer, while others
        employ look-ahead methods to predict malformed simulation inputs. These 
        methods fail to realistically model the process by which utilities, 
        governments, and other stakeholders actually make facility deployment 
        decisions.  

        To match the natural form of the problem, \gls{NFC} simulators must 
        bring demand and deployment decisions into the dynamics of the 
        simulation logic. The authors seek to identify the most flexible, 
        general, and performant algorithms for this purpose.  To inform this 
        pursuit, a review was conducted of current \gls{NFC} simulation 
        tools, to determine the current capabilites for a demand-driven scenario 
        formulation. This paper summarizes that review.
\end{abstract}
