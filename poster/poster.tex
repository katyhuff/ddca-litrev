%%%%%%%%%%%%%%%%%%%%%%%%%%%%%%%%%%%%%%%%%
% Jacobs Landscape Poster
% LaTeX Template
% Version 1.1 (14/06/14)
%
% Created by:
% Computational Physics and Biophysics Group, Jacobs University
% https://teamwork.jacobs-university.de:8443/confluence/display/CoPandBiG/LaTeX+Poster
% 
% Further modified by:
% Nathaniel Johnston (nathaniel@njohnston.ca)
%
% This template has been downloaded from:
% http://www.LaTeXTemplates.com
%
% License:
% CC BY-NC-SA 3.0 (http://creativecommons.org/licenses/by-nc-sa/3.0/)
%
%%%%%%%%%%%%%%%%%%%%%%%%%%%%%%%%%%%%%%%%%

%----------------------------------------------------------------------------------------
%	PACKAGES AND OTHER DOCUMENT CONFIGURATIONS
%----------------------------------------------------------------------------------------

\documentclass[final]{beamer}

\usepackage[scale=1.24]{beamerposter} % Use the beamerposter package for laying out the poster
\usetheme{confposter} % Use the confposter theme supplied with this template

\setbeamercolor{block title}{fg=ngreen,bg=white} % Colors of the block titles
\setbeamercolor{block body}{fg=black,bg=white} % Colors of the body of blocks
\setbeamercolor{block alerted title}{fg=white,bg=dblue!70} % Colors of the highlighted block titles
\setbeamercolor{block alerted body}{fg=black,bg=dblue!10} % Colors of the body of highlighted blocks
% Many more colors are available for use in beamerthemeconfposter.sty

%-----------------------------------------------------------
% Define the column widths and overall poster size
% To set effective sepwid, onecolwid and twocolwid values, first choose how many columns you want and how much separation you want between columns
% In this template, the separation width chosen is 0.024 of the paper width and a 4-column layout
% onecolwid should therefore be (1-(# of columns+1)*sepwid)/# of columns e.g. (1-(4+1)*0.024)/4 = 0.22
% Set twocolwid to be (2*onecolwid)+sepwid = 0.464
% Set threecolwid to be (3*onecolwid)+2*sepwid = 0.708

\newlength{\sepwid}
\newlength{\onecolwid}
\newlength{\twocolwid}
\newlength{\threecolwid}
\setlength{\paperwidth}{48in} % A0 width: 46.8in
\setlength{\paperheight}{36in} % A0 height: 33.1in
\setlength{\sepwid}{0.024\paperwidth} % Separation width (white space) between columns
\setlength{\onecolwid}{0.22\paperwidth} % Width of one column
\setlength{\twocolwid}{0.464\paperwidth} % Width of two columns
\setlength{\threecolwid}{0.708\paperwidth} % Width of three columns
\setlength{\topmargin}{-0.5in} % Reduce the top margin size
%-----------------------------------------------------------

\usepackage{graphicx}  % Required for including images
\newcommand{\Cyclus}{\textsc{Cyclus}\xspace}%

\usepackage{tabularx}
\newcolumntype{b}{X}
\newcolumntype{s}{>{\hsize=.5\hsize}X}
\newcolumntype{m}{>{\hsize=.75\hsize}X}
\newcolumntype{z}{>{\hsize=.65\hsize}X}

\usepackage{booktabs} % Top and bottom rules for tables
\usepackage{xspace}

\usepackage{tikz}
\usetikzlibrary{positioning, arrows, decorations, shapes }
% Define block styles
\tikzstyle{decision} = [diamond, draw, fill=blue!20, 
text width=4.5em, text badly centered, node distance=3cm, inner sep=0pt]


\tikzstyle{block} = [rectangle, draw, text centered, fill=blue!20]
\tikzstyle{line} = [draw, -latex']
\tikzstyle{cloud} = [draw, ellipse,fill=red!20, node distance=6em,
minimum height=2em]



\usetikzlibrary{shapes.multipart}
\usetikzlibrary{positioning}


\setbeamertemplate{bibliography item}[text]

\bibliographystyle{abbrv}
\bibliography{poster}


%----------------------------------------------------------------------------------------
%	TITLE SECTION 
%----------------------------------------------------------------------------------------

\title{Current Status of Predictive Transition Capability in Fuel Cycle Simulation} % Poster title

\author{Kathryn D. Huff$^1$, Jin whan bae$^1$, Robert R. Flanagan$^2$, Anthony M. Scopatz$^2$}
\institute{$^1$University of Illinios at Urbana-Champaign, Department of Nuclear, Plasma, and Radiological Engineering, Urbana, IL 61801\\
	$^2$University of South Carolina, Nuclear Engineering Program, Department of Mechanical Engineering, Columbia, SC 29208}
%----------------------------------------------------------------------------------------

\begin{document}

\addtobeamertemplate{block end}{}{\vspace*{2ex}} % White space under blocks
\addtobeamertemplate{block alerted end}{}{\vspace*{2ex}} % White space under highlighted (alert) blocks

\setlength{\belowcaptionskip}{2ex} % White space under figures
\setlength\belowdisplayshortskip{2ex} % White space under equations

\begin{frame}[t] % The whole poster is enclosed in one beamer frame

\begin{columns}[t] % The whole poster consists of three major columns, the second of which is split into two columns twice - the [t] option aligns each column's content to the top

\begin{column}{\sepwid}\end{column} % Empty spacer column

\begin{column}{\onecolwid} % The first column

%----------------------------------------------------------------------------------------
%	OBJECTIVES
%----------------------------------------------------------------------------------------

\begin{alertblock}{Objectives}

Identify flexible, general, and performant algorithms available for application to simulating
demand-driven deployment of nuclear fuel cycle facility capacity in a fuel cycle simulator.
\begin{itemize}
	\item Review current Nuclear Fuel Cycle simulation tools to determine their current capabilities.
	\item Investigate promising algorithms.
	\item Applications of such algorithms in other domains (economics, industrial engineering).  
\end{itemize}

\end{alertblock}

%----------------------------------------------------------------------------------------
%	INTRODUCTION
%----------------------------------------------------------------------------------------

\begin{block}{Introduction}

\begin{figure}
	\centering
	\scalebox{0.8}{
		\begin{tikzpicture}[    >=stealth,
		node distance=6cm,
		on grid,
		align=center,
		auto]
		% Place nodes
		\node [block] (of) {\textbf{NFC Simulation = Objective Functions}};
		\node [cloud, below of=of] (dem) {\texttt{DEMAND} \\ E.g. Achieve 1\% growth for total electricity};
		\node [cloud, below of=dem] (con) {\texttt{CONSTRAINT} \\E.g. Reprocessing available from 2025 };
		
		\draw[->, thick] (of) -- (dem);
		\draw[->, thick] (dem) -- (con);
		
		\end{tikzpicture}
		
	}
\end{figure}
Need for : \\ Demand Repsonsive Deployment Decisions
\begin{figure}
	\centering
	\scalebox{0.80}{
		\begin{tikzpicture}[    >=stealth,
		node distance=3cm,
		on grid,
		align=center,
		auto]
		% Place nodes
		\node [block] (rea) {{\LARGE \texttt{Reactor}}};
		\node [cloud, below of=rea] (d1) {1. Demand: \textbf{Fuel}};
		\node [cloud, below of=d1] (s1) {2. Deploy \texttt{Enrichment} Facility \\ to meet fuel demand};
		\node [cloud, below of=s1] (ff) {3. Deploy \texttt{Mills} to meet UF6 demand};
		\node [cloud, below of=ff] (d2) {4. Deploy \texttt{Mines} to meet uranium demand};
		
		
		\draw[->, thick] (rea) -- (d1);
		\draw[->, thick] (d1) -- (s1);
		\draw[->, thick] (s1) -- (ff);
		\draw[->, thick] (ff) -- (d2);
		\end{tikzpicture}
		
	}
	\caption{Dynamic Deployment to meet fuel demand}
\end{figure}


\end{block}

\end{column} % End of the first column

\begin{column}{\sepwid}\end{column} % Empty spacer column

\begin{column}{\twocolwid} % Begin a column which is two columns wide (column 2)

\begin{columns}[t,totalwidth=\twocolwid] % Split up the two columns wide column

\begin{column}{\onecolwid}\vspace{-.6in} % The first column within column 2 (column 2.1)

%----------------------------------------------------------------------------------------
%	MATERIALS
%----------------------------------------------------------------------------------------

\begin{block}{Methods}
	A meta-review of previous Nuclear Fuel Cycle gap analyses helped to identify the existing
	simulators and their capabilities. The strongest comparisons of transition scenario
	code capabilities are found in \cite{boucher_international_2010}, \cite{brown_identification_2016} and \cite{mccarthy_benchmark_2012}.
	In \cite{carre_overview_2009} and \cite{hoffman_expanded_2016}, the ability of
	individual simulators to conduct transition scenarios is addressed, but not the
	flexibility and performance of their varying algorithms for this capability.
	

\end{block}

%----------------------------------------------------------------------------------------

\end{column} % End of column 2.1

\begin{column}{\onecolwid}\vspace{-.6in} % The second column within column 2 (column 2.2)

%----------------------------------------------------------------------------------------
%	METHODS
%----------------------------------------------------------------------------------------

\begin{block}{Successful Applications}
{\large 	\begin{itemize}
		\item Lumber mills \cite{yanex_agent-based_2009}
		\item weather-responsive building efficiency \cite{gonzalez_detailed_2002, kusiak_data-driven_2010}
		\item airline routing logistics \cite{shebalov_practical_2009}
	\end{itemize}
	}
These algorithms have succeeded in other domains for similar classes of problems.
Accordingly, their potential in nuclear fuel cycle analysis ins promising.
\end{block}

%----------------------------------------------------------------------------------------

\end{column} % End of column 2.2

\end{columns} % End of the split of column 2 - any content after this will now take up 2 columns width

%----------------------------------------------------------------------------------------
%	IMPORTANT RESULT
%----------------------------------------------------------------------------------------

%----------------------------------------------------------------------------------------

\begin{columns}[t,totalwidth=\twocolwid] % Split up the two columns wide column again

\begin{column}{\onecolwid} % The first column within column 2 (column 2.1)

%----------------------------------------------------------------------------------------
%	MATHEMATICAL SECTION
%----------------------------------------------------------------------------------------

\begin{block}{Current Strategies}
	
	\begin{itemize}
		\item \textbf{manual (MAN):} The user `guesses' future deployment of reactor and facility.
		\item \textbf{proportional (PROP):} Deployment of fuel cycle facilities is in 
		direct proportion with reactor deployments .
		\item \textbf{constrained reactor deployment (CONST):} Deployment of reactors is 
		constrained by the existing and projected feedstock amounts.
		\item \textbf{predictive (PRED):} The simulator projects feedstock needs of 
		current and future deployed reactors based on other heuristics 
		and look-ahead predictions. 
		\item \textbf{Demand-Driven (D-D):} The simulator deploys facilities according
		to demand
	\end{itemize}

	\begin{table}
		\centering
		\scalebox{.85}{
		\begin{tabularx}{\textwidth}{mmss}
			\hline 
			\textbf{Simulator} & \textbf{Institution} & \textbf{Reactor} & \textbf{Facility}\\ \hline
			CAFCA \cite{guerin_impact_2009} & MIT    & MAN & MAN \\
			CLASS \cite{mouginot_class_2012} & CNRS/IRSN & MAN & MAN \\
			COSI \cite{coquelet-pascal_cosi6:_2015,boucher_international_2010} & CEA & 
			D-D & PRED \\
			\Cyclus \cite{huff_fundamental_2016} & UW &  D-D & MAN \\ 
			DESAE \cite{boucher_international_2010}& Rosatom &  MAN & MAN \\
			DANESS \cite{van_den_durpel_daness:_2006} & ANL & D-D & PROP \\
			DYMOND \cite{park_modeling_2003}& ANL &  D-D & PRED \\
			Evolcode\cite{boucher_international_2010} & CIEMAT & D-D & MAN\\
			FAMILY \cite{boucher_international_2010}&  IAEA &  D-D & PRED \\
			MARKAL \cite{feng_standardized_2016}& BNL &  D-D & MAN\\
			NFCSim \cite{schneider_nfcsim:_2005}& LANL &  D-D & PRED \\
			NGSAM \cite{aubin_development_2013} & ORNL & NONE & MAN \\
			{\small NUWASTE \cite{garrick_nuclear_2011}}& NWTRB &  MAN & MAN \\
			ORION \cite{feng_standardized_2016} & NNL & CONST  & PROP\\
			VISION \cite{feng_standardized_2016,boucher_international_2010}& INL  & D-D & PROP \\
			VISTA \cite{iaea_nuclear_2007}& IAEA & D-D & PROP \\ \hline
		\end{tabularx}}
		\caption{Simulators, categorized by their reactor and fuel cycle 
			facility deployment strategies.}
		\label{tam:sim-deploy}
	\end{table}
	
\end{block}

%----------------------------------------------------------------------------------------

\end{column} % End of column 2.1

\begin{column}{\onecolwid} % The second column within column 2 (column 2.2)

%----------------------------------------------------------------------------------------
%	RESULTS
%----------------------------------------------------------------------------------------

\begin{block}{Promising Algorithms}
 \begin{itemize}
 	\setlength\itemsep{1.5em}
	\item \textbf{{\Large Non-Optimizing (NO)}}
	\begin{itemize}
		\setlength\itemsep{1em}
		\item {\large Predict based on historical supply-demand data}
		\item{\large  Does not attempt to meet demand optimally}
		\item {\large Fast execution time with limited precision}
		\item {\large E.g. Autoregressive Moving Average (ARMA), Autoregressive Conditional Heteroskedastic (ARCH)}
	\end{itemize}
	\item \textbf{{\Large Deterministic Optimizing (DO)}}
	\begin{itemize}
		\setlength\itemsep{1em}
		\item{\large  Minimize or Maximize an objective function with set of constraints}
		\item {\large Replicable}
		\item{\large  E.g. Global Change Assessment Model (GCAM), MARKet and ALlocation (MARKAL)}
	\end{itemize}
	\item \textbf{{\Large Stochastic Optimizing (SO)}}
	\begin{itemize}
		\setlength\itemsep{1em}
		\item {\large Probabilistic search into the objective function or constraint models}
		\item {\large Uncertainty in addition to mean}
		\item {\large Uses random samples from probability distributions}
		\item {\large E.g. Markov Switching-Model, Gaussian Process Regression}
	\end{itemize}
\end{itemize}


\end{block}



%----------------------------------------------------------------------------------------

\end{column} % End of column 2.2

\end{columns} % End of the split of column 2

\end{column} % End of the second column

\begin{column}{\sepwid}\end{column} % Empty spacer column

\begin{column}{\onecolwid} % The third column

%----------------------------------------------------------------------------------------
%	CONCLUSION
%----------------------------------------------------------------------------------------

\begin{block}{Conclusion}
The review concludes that fuel cycle simulation tools approach scenario objective 
functions in various ways.
\begin{itemize}
	\item wrapping in external optimizer
	\item look-ahead methods to predict input validity
\end{itemize} 
These methods fail to realistically model the process by which utilities, governments,
and other stakeholders actually make facility deployment decisions.
Dynamic, demand-driven facility deployment may be enabled by algorithms in use in
other fields. Different deployment models differ in terms of both compute
speed, flexibility (in terms of the range of scenarios capably
simulated), and robustness (in terms of consistent fidelity of the modeling results).
Finally, current NFC simulators may more flexibly support demand-driven deployment
through incorporation of non-optimizing algorithms such as ARMA \cite{woodard_stationarity_2011} and ARCH \cite{li_kernal_2016},
deterministically optimizing methods such as those collected in GCAM \cite{edmonds_advanced_1994} and
MARKAL \cite{fishbone_markal_1981}, or stochastic optimization techniques such as Markov Switching Models \cite{ansari_predicting_2015}.
\end{block}

%----------------------------------------------------------------------------------------
%	ACKNOWLEDGEMENTS
%----------------------------------------------------------------------------------------

\setbeamercolor{block title}{fg=red,bg=white} % Change the block title color

\begin{block}{Acknowledgements}

This research was performed using funding received
from the DOE Office of Nuclear Energy's Nuclear Energy
University Programs under award number 16-10512.

\end{block}


%----------------------------------------------------------------------------------------
%	CONTACT INFORMATION
%----------------------------------------------------------------------------------------

\setbeamercolor{block alerted title}{fg=black,bg=norange} % Change the alert block title colors
\setbeamercolor{block alerted body}{fg=black,bg=white} % Change the alert block body colors

\begin{alertblock}{Contact Information}

\begin{itemize}
	
	\item Web: \href{arfc.github.io}{arfc.github.io}
	\item Email: \href{mailto:jbae11@illinois.edu}{jbae11@illinois.edu}
	\item Phone: +1 (217) 377-5784
\end{itemize}

\end{alertblock}

\begin{center}
\begin{tabular}{ccc}
\includegraphics[width=0.3\linewidth]{logo.png} & \includegraphics[width=0.3\linewidth]{ergs_logo.png} & \includegraphics[width=0.3\linewidth]{neup.png}
\end{tabular}
\end{center}

%----------------------------------------------------------------------------------------

\end{column} % End of the third column

\end{columns} % End of all the columns in the poster

\end{frame} % End of the enclosing frame

\end{document}
