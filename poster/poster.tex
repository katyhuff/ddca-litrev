%%%%%%%%%%%%%%%%%%%%%%%%%%%%%%%%%%%%%%%%%
% Jacobs Landscape Poster
% LaTeX Template
% Version 1.1 (14/06/14)
%
% Created by:
% Computational Physics and Biophysics Group, Jacobs University
% https://teamwork.jacobs-university.de:8443/confluence/display/CoPandBiG/LaTeX+Poster
% 
% Further modified by:
% Nathaniel Johnston (nathaniel@njohnston.ca)
%
% This template has been downloaded from:
% http://www.LaTeXTemplates.com
%
% License:
% CC BY-NC-SA 3.0 (http://creativecommons.org/licenses/by-nc-sa/3.0/)
%
%%%%%%%%%%%%%%%%%%%%%%%%%%%%%%%%%%%%%%%%%

%----------------------------------------------------------------------------------------
%	PACKAGES AND OTHER DOCUMENT CONFIGURATIONS
%----------------------------------------------------------------------------------------

\documentclass[final]{beamer}

\usepackage[scale=1.0]{beamerposter} % Use the beamerposter package for laying out the poster
\usetheme{confposter} % Use the confposter theme supplied with this template

\setbeamercolor{block title}{fg=dblue!80,bg=white} % Colors of the block titles
\setbeamercolor{block body}{fg=black,bg=white} % Colors of the body of blocks
\setbeamercolor{block alerted title}{fg=white,bg=dblue!70} % Colors of the highlighted block titles
\setbeamercolor{block alerted body}{fg=black,bg=dblue!10} % Colors of the body of highlighted blocks
% Many more colors are available for use in beamerthemeconfposter.sty

%-----------------------------------------------------------
% Define the column widths and overall poster size
% To set effective sepwid, onecolwid and twocolwid values, first choose how many columns you want and how much separation you want between columns
% In this template, the separation width chosen is 0.024 of the paper width and a 4-column layout
% onecolwid should therefore be (1-(# of columns+1)*sepwid)/# of columns e.g. (1-(4+1)*0.024)/4 = 0.22
% onecolwid should therefore be (1-(# of columns+1)*sepwid)/# of columns e.g. 
% (1-(3+1)*0.025)/3 = 0.3
% Set twocolwid to be (2*onecolwid)+sepwid = 0.464
% Set threecolwid to be (3*onecolwid)+2*sepwid = 0.708

\newlength{\sepwid}
\newlength{\onecolwid}
\newlength{\twocolwid}
\newlength{\threecolwid}
\setlength{\paperwidth}{36in} % A0 width: 46.8in
\setlength{\paperheight}{48in} % A0 height: 33.1in
\setlength{\textwidth}{34in} % A0 width: 46.8in
\setlength{\textheight}{46in} % A0 height: 33.1in
\setlength{\sepwid}{0.025\paperwidth} % Separation width (white space) between columns
\setlength{\onecolwid}{0.3\paperwidth} % Width of one column
\setlength{\twocolwid}{0.625\paperwidth} % Width of two columns
\setlength{\threecolwid}{0.95\paperwidth} % Width of three columns
\setlength{\topmargin}{-0.5in} % Reduce the top margin size
%-----------------------------------------------------------

\usepackage{graphicx}  % Required for including images
\newcommand{\Cyclus}{\textsc{Cyclus}\xspace}%

\usepackage{tabularx}
\newcolumntype{b}{X}
\newcolumntype{s}{>{\hsize=.5\hsize}X}
\newcolumntype{m}{>{\hsize=.75\hsize}X}
\newcolumntype{z}{>{\hsize=.65\hsize}X}

\usepackage{booktabs} % Top and bottom rules for tables
\usepackage{xspace}

\usepackage{tikz}
\usetikzlibrary{positioning, arrows, decorations, shapes }
% Define block styles
\tikzstyle{decision} = [diamond, draw, fill=blue!20, 
text width=4.5em, text badly centered, node distance=3cm, inner sep=0pt]


\tikzstyle{block} = [rectangle, draw, text centered, fill=blue!20]
\tikzstyle{line} = [draw, -latex']
\tikzstyle{cloud} = [draw, ellipse,fill=red!20, node distance=6em,
minimum height=2em]



\usetikzlibrary{shapes.multipart}
\usetikzlibrary{positioning}


\setbeamertemplate{bibliography item}[text]

%----------------------------------------------------------------------------------------
%	TITLE SECTION 
%----------------------------------------------------------------------------------------

\title{Current Status of Predictive Transition Capability in Fuel Cycle Simulation} % Poster title

\author{Kathryn D. Huff$^1$, Jin whan bae$^1$, Robert R. Flanagan$^2$, Anthony M. Scopatz$^2$}
\institute{$^1$University of Illinios at Urbana-Champaign, Department of Nuclear, Plasma, and Radiological Engineering, Urbana, IL 61801\\
	$^2$University of South Carolina, Nuclear Engineering Program, Department of Mechanical Engineering, Columbia, SC 29208}
%----------------------------------------------------------------------------------------

\begin{document}

\addtobeamertemplate{block end}{}{\vspace*{2ex}} % White space under blocks
\addtobeamertemplate{block alerted end}{}{\vspace*{2ex}} % White space under highlighted (alert) blocks

\setlength{\belowcaptionskip}{2ex} % White space under figures
\setlength\belowdisplayshortskip{2ex} % White space under equations

\begin{frame}[t] % The whole poster is enclosed in one beamer frame

\begin{columns}[t,totalwidth=\threecolwid] % The whole poster consists of three major columns, the second of which is split into two columns twice - the [t] option aligns each column's content to the top

\begin{column}{\sepwid}\end{column} % Empty spacer column

\begin{column}{\onecolwid} % The first column

%----------------------------------------------------------------------------------------
%	OBJECTIVES
%----------------------------------------------------------------------------------------

\begin{alertblock}{Objectives}

Identify flexible, general, and performant algorithms available for application to simulating
demand-driven deployment of nuclear fuel cycle facility capacity in a fuel cycle simulator.
\begin{itemize}
	\item Review nuclear fuel cycle simulator state-of-the-art.
	\item Investigate promising prediction algorithms.
	\item Identify algorithms successful in other domains.
\end{itemize}

\end{alertblock}

%----------------------------------------------------------------------------------------
%	INTRODUCTION
%----------------------------------------------------------------------------------------

\begin{block}{Introduction}

\begin{figure}
	\centering
	\scalebox{0.8}{
		\begin{tikzpicture}[    >=stealth,
		node distance=6cm,
		on grid,
		align=center,
		auto]
		% Place nodes
		\node [block] (of) {\textbf{NFC Simulation = Objective Functions}};
		\node [cloud, below of=of] (dem) {\texttt{DEMAND} \\ E.g. Achieve 1\% growth for total electricity};
		\node [cloud, below of=dem] (con) {\texttt{CONSTRAINT} \\E.g. Reprocessing available from 2025 };
		
		\draw[->, thick] (of) -- (dem);
		\draw[->, thick] (dem) -- (con);
		
		\end{tikzpicture}
		
	}
	\caption{Nuclear fuel cycle scenarios are 
constrained objective functions.}
\end{figure}

\begin{figure}
	\centering
	\scalebox{0.80}{
		\begin{tikzpicture}[    >=stealth,
		node distance=3cm,
		on grid,
		align=center,
		auto]
		% Place nodes
		\node [block] (rea) {{\LARGE \texttt{Reactor}}};
		\node [cloud, below of=rea] (d1) {1. Demand: \textbf{Fuel}};
		\node [cloud, below of=d1] (s1) {2. Deploy \texttt{Enrichment} Facility \\ to meet fuel demand};
		\node [cloud, below of=s1] (ff) {3. Deploy \texttt{Mills} to meet UF6 demand};
		\node [cloud, below of=ff] (d2) {4. Deploy \texttt{Mines} to meet uranium demand};
		
		
		\draw[->, thick] (rea) -- (d1);
		\draw[->, thick] (d1) -- (s1);
		\draw[->, thick] (s1) -- (ff);
		\draw[->, thick] (ff) -- (d2);
		\end{tikzpicture}
		
	}
	\caption{Dynamic deployment seeks to meet fuel demand.}
\end{figure}
\end{block}

%----------------------------------------------------------------------------------------
%	MATERIALS
%----------------------------------------------------------------------------------------

\begin{block}{Methods}
        A review of nuclear fuel cycle simulators (Table \ref{tab:sim-deploy}) and previous Nuclear Fuel 
        Cycle gap analyses 
        \cite{boucher_international_2010, brown_identification_2016, 
        mccarthy_benchmark_2012,carre_overview_2009,hoffman_expanded_2016}
        distinguished existing simulators with regard to facility deployment 
        and transition scenario capabilities. Capabilities were categorized as:
	
	\begin{itemize}
		\item \textbf{manual (MAN):} The user `guesses' future deployment of reactor and facility.
		\item \textbf{proportional (PROP):} Deployment of fuel cycle facilities is in 
		direct proportion with reactor deployments .
		\item \textbf{constrained reactor deployment (CONST):} Deployment of reactors is 
		constrained by the existing and projected feedstock amounts.
		\item \textbf{predictive (PRED):} The simulator projects feedstock needs of 
		current and future deployed reactors based on other heuristics 
		and look-ahead predictions. 
		\item \textbf{Demand-Driven (D-D):} The simulator deploys facilities according
		to demand
	\end{itemize}

	\begin{table}
		\centering
		\begin{tabularx}{\textwidth}{mmss}
			\hline 
			\textbf{Simulator} & \textbf{Institution} & \textbf{Reactor} & \textbf{Facility}\\ \hline
			CAFCA \cite{guerin_impact_2009} & MIT    & MAN & MAN \\
			CLASS \cite{mouginot_class_2012} & CNRS/IRSN & MAN & MAN \\
			COSI \cite{coquelet-pascal_cosi6:_2015,boucher_international_2010} & CEA & 
			D-D & PRED \\
			\Cyclus \cite{huff_fundamental_2016} & UW &  D-D & MAN \\ 
			DESAE \cite{boucher_international_2010}& Rosatom &  MAN & MAN \\
			DANESS \cite{van_den_durpel_daness:_2006} & ANL & D-D & PROP \\
			DYMOND \cite{park_modeling_2003}& ANL &  D-D & PRED \\
			Evolcode\cite{boucher_international_2010} & CIEMAT & D-D & MAN\\
			FAMILY \cite{boucher_international_2010}&  IAEA &  D-D & PRED \\
			MARKAL \cite{feng_standardized_2016}& BNL &  D-D & MAN\\
			NFCSim \cite{schneider_nfcsim:_2005}& LANL &  D-D & PRED \\
			NGSAM \cite{aubin_development_2013} & ORNL & NONE & MAN \\
			NUWASTE \cite{garrick_nuclear_2011} & NWTRB &  MAN & MAN \\
			ORION \cite{feng_standardized_2016} & NNL & CONST  & PROP\\
			VISION \cite{feng_standardized_2016,boucher_international_2010}& INL  & D-D & PROP \\
			VISTA \cite{iaea_nuclear_2007}& IAEA & D-D & PROP \\ \hline
		\end{tabularx}
		\caption{Simulators, categorized by their reactor and fuel cycle 
			facility deployment strategies.}
		\label{tab:sim-deploy}
	\end{table}
	
\end{block}

%----------------------------------------------------------------------------------------

\end{column} % End of the first column

\begin{column}{\sepwid}\end{column} % Empty spacer column


%----------------------------------------------------------------------------------------

\begin{column}{\onecolwid} % The second column

%----------------------------------------------------------------------------------------
%	IMPORTANT RESULT
%----------------------------------------------------------------------------------------


%----------------------------------------------------------------------------------------
%	RESULTS
%----------------------------------------------------------------------------------------
\begin{block}{Promising Algorithms}\end{block}

        \begin{alertblock}{Non-Optimizing (NO)}
	\begin{itemize}
		\item {\large Predict based on historical supply-demand data}
		\item{\large  Does not attempt to meet demand optimally}
		\item {\large Fast execution time with limited precision}
		\item {\large E.g. Autoregressive Moving Average (ARMA), Autoregressive Conditional Heteroskedastic (ARCH)}
	\end{itemize}
        \end{alertblock}
        
        \begin{alertblock}{Deterministic Optimizing (DO)}
	\begin{itemize}
		\item{\large  Optimizes an objective function with set of constraints}
		\item {\large Replicable}
		\item{\large  E.g. Global Change Assessment Model (GCAM), MARKet and ALlocation (MARKAL)}
	\end{itemize}
        \end{alertblock}

        \begin{alertblock}{Stochastic Optimizing (SO)}
	\begin{itemize}
%		\setlength\itemsep{1em}
		\item {\large Probabilistic search into the objective function or constraint models}
		\item {\large Uncertainty in addition to mean}
		\item {\large Uses random samples from probability distributions}
		\item {\large E.g. Markov Switching-Model, Gaussian Process Regression}
	\end{itemize}
        \end{alertblock}

\begin{block}{Successful Applications}
These algorithms have succeeded for similar classes of problems in other domains, such as:
\begin{itemize}
		\item lumber mill operations \cite{yanez_agent-based_2009}
		\item weather-responsive building efficiency \cite{gonzalez_detailed_2002, kusiak_data-driven_2010}
		\item airline routing logistics \cite{shebalov_practical_2009}
	\end{itemize}
Accordingly, their potential in nuclear fuel cycle analysis is promising.
\end{block}

%----------------------------------------------------------------------------------------
%	CONCLUSION
%----------------------------------------------------------------------------------------

\begin{block}{Conclusion}
The review concludes that fuel cycle simulation tools approach scenario objective 
functions by
\begin{itemize}
	\item wrapping in an external optimizer
	\item or predicting deployment strategy with look-ahead methods.
\end{itemize} 
 \textbf{Deployment models differ in terms of compute
speed, flexibility} (in terms of the range of scenarios capably
simulated), \textbf{and robustness} (in terms of consistent fidelity of the modeling results).
Finally, current NFC simulators may more flexibly support demand-driven deployment
through incorporation of non-optimizing algorithms such as ARMA \cite{woodard_stationarity_2011} and ARCH \cite{li_kernel_2016},
deterministically optimizing methods such as those collected in GCAM \cite{edmonds_advanced_1994} and
MARKAL \cite{fishbone_markal_1981}, or stochastic optimization techniques such as Markov Switching Models \cite{ansari_predicting_2015}. Such algorithms succeed in other fields.
\end{block}

%----------------------------------------------------------------------------------------
%	ACKNOWLEDGEMENTS
%----------------------------------------------------------------------------------------

\setbeamercolor{block title}{fg=norange,bg=white} % Change the block title color

\begin{block}{Acknowledgements}

This research was performed using funding received
from the DOE Office of Nuclear Energy's Nuclear Energy
University Programs under award number 16-10512.

\vspace{10mm}
\begin{center}
\begin{tabular}{ccc}
\includegraphics[width=0.3\linewidth]{logo.png} & \includegraphics[width=0.3\linewidth]{ergs_logo.png} & \includegraphics[width=0.3\linewidth]{neup.png}
\end{tabular}
\end{center}


\end{block}

%----------------------------------------------------------------------------------------
%	CONTACT INFORMATION
%----------------------------------------------------------------------------------------

\setbeamercolor{block alerted title}{fg=black,bg=norange} % Change the alert block title colors
\setbeamercolor{block alerted body}{fg=black,bg=white} % Change the alert block body colors



\begin{alertblock}{Contact Information}
\setbeamercolor{block title}{fg=norange,bg=white} % Change the block title color
\begin{itemize}
	
	\item Web: \href{arfc.github.io}{arfc.github.io}
	\item Email: \href{mailto:jbae11@illinois.edu}{jbae11@illinois.edu}
	\item Phone: +1 (217) 377-5784
\end{itemize}

\end{alertblock}

%----------------------------------------------------------------------------------------

\end{column} % End of column 2

\begin{column}{\sepwid}\end{column} % Empty spacer column

\begin{column}{\onecolwid} % The third column


\begin{block}{References}

        {\footnotesize\bibliographystyle{abbrv} 
        \bibliography{poster}}
\end{block}


%----------------------------------------------------------------------------------------



\end{column} % End of the third column

\end{columns} % End of all the columns in the poster

\end{frame} % End of the enclosing frame

\end{document}
\begin{column}{\sepwid}\end{column} % Empty spacer column
